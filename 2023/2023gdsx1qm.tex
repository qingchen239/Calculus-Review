\documentclass[12pt, addpoints]{exam}
\usepackage{xeCJK}
\setCJKmainfont{Noto Serif CJK TC}[
  AutoFakeBold=2,
  BoldFont=Noto Serif CJK TC Bold,
  ItalicFont=AR PL UKai TW,
  BoldItalicFont=AR PL UKai TW,
  BoldItalicFeatures={FakeBold=2}
]
\usepackage{amsfonts,amssymb,amsmath, amsthm}
\usepackage{graphicx}
\usepackage{systeme}
\usepackage{pgf,tikz,pgfplots}
\pgfplotsset{compat=1.15}
\usepgfplotslibrary{fillbetween}
\usepackage{mathrsfs}
\usetikzlibrary{arrows}
\usetikzlibrary{calc}

\usepackage{multicol}

\def \important{\textbf{(重點)}}
\def \unimportant{\textit{(非重點)}}
\def \hint#1{(\textit{#1})}
\def \d{\mathrm{d}}
\def \arccot{\mathrm{arccot}}

\pointname{分}
\boxedpoints
\pointsinrightmargin
\marginpointname{\textit{分}}

% \renewcommand{\thepartno}{\arabic{partno}}
% \newcommand{\partlabel}{(\thepartno)}

\pagestyle{headandfoot}

\firstpageheader{《高等數學\textit{(上)}》期末複習題(滿分\numpoints\ 分)\\ 鏡州商貿學院\textit{(新圩)}}{}{姓名:\underline{\hspace{2.5in}}}
%\firstpageheadrule

\runningheader{鏡州商貿學院\textit{(新圩)}}{}{第 \thepage\ 頁,共 \numpages 頁}
\runningheadrule

\firstpagefooter{}{}{}
\runningfooter{}{}{}


\begin{document}

\begin{center}
\fbox{\fbox{\parbox{6in}{%\centering
\textbf{免責聲明:}\\本補習為鏡州商貿學院\textit{(新圩)}部分學生之\textbf{自發行為},與其他任何學校、任何老師或任何同學無關。本課程\textbf{不佔學時,不影響平時成績,不附贈二課學分},不強制同學參加,也\textbf{不保證參加同學一定能夠及格}。如果本補習使用之複習題與任何學校之教材或考試題目雷同,那\textbf{太正常了}。因為《高等數學》一共就這麼幾個知識點,變來變去也變不出多少花樣。你讓我上哪兒找那麼多新題型去?
}}}
\end{center}

\fullwidth{注:在本複習題中,帶有\important 標註的題目為該小節中的關鍵概念及重點題目,這些題目在期末考試中往往經常出現,並且稍加學習便很容易掌握,是我們複習的重點。帶有\unimportant 標註的題目為非重點題目,這些題目在期末考試中很少出現,或者學習難度很高,不是我們複習的重點。不帶有標註的題目為普通題目,它們在期末考試中出現的可能性和學習難度一般。}

\section{函數、極限、連續}

\subsection{函數極限與數列極限的計算}

\begin{questions} %------------------------------------------

\question
\important 默寫常用的麥克勞林公式。
\begin{parts}
    \part
    $\displaystyle e^x =$
    \vspace{\stretch{2}}
    \part
    $\displaystyle \sin{x} =$
    \vspace{\stretch{2}}
    \part
    $\displaystyle \cos{x} =$
    \vspace{\stretch{2}}
    \part
    $\displaystyle \ln{(1+x)} =$
    \vspace{\stretch{2}}
\end{parts}

\newpage %---------------------------------------------------

\question
\important 極限的計算方法:
\begin{description}
    \item[麥克勞林公式:] 
        \begin{enumerate}
            \item
            沒有$0$的直接帶進去算;
            \item
            有$0$的想辦法把$0$顯現出來;
            \item
            帶麥克勞林公式化成多項式;
            \item
            把$0$約掉,得到結果。
        \end{enumerate}
    \item[麥克勞林公式的用法:] 乘除帶一項;加減帶多項,帶到與分子/分母同次為止。
    \item[洛必達法則:] 若[\textit{略}],則$\displaystyle \lim_{x \to x_0} \frac{f(x)}{g(x)} = \lim_{x \to x_0} \frac{f'(x)}{g'(x)}$
    \item[$1$加$0$的次方:] 對於$\displaystyle (1+0)^?$形式的極限,要翻成$\displaystyle e^{? \ln{(1+0)}}$的型式再計算。
\end{description}
\vspace{\stretch{1}}

\question[1]
下列極限正確的是
\begin{choices}
    \choice $\displaystyle \lim_{x \to 0^-} e^{\frac{1}{x}}=0$
    \choice $\displaystyle \lim_{x \to 0^+} e^{\frac{1}{x}}=0$
    \choice $\displaystyle \lim_{x \to 0} (1+\cos{x})^{\sec x}=e$
    \choice $\displaystyle \lim_{x \to \infty} (1+x)^{\frac{1}{x}}=e$
\end{choices}
\vspace{\stretch{1}}

\newpage %---------------------------------------------------

\question
計算下列極限。
\begin{parts}
    \part[1]
    $\displaystyle \lim_{x \to \sqrt{3}} \frac{x^2-2}{x^4+x^2+1}$
    \vspace{\stretch{2}}
    \part[1]
    $\displaystyle \lim_{x \to 1^-} \frac{x-1}{|x-1|}$
    \vspace{\stretch{2}}
    \part[1]
    $\displaystyle \lim_{x \to 1} \left( \frac{1}{1-x}-\frac{3}{1-x^3} \right)$
    \vspace{\stretch{2}}
    \part[1]
    $\displaystyle \lim_{x \to \infty} \frac{(2x-3)^{20} (3x+2)^{30}}{(5x+1)^{50}}$
    \vspace{\stretch{2}}
    \part[1]
    $\displaystyle \lim_{x \to 0} \frac{\sin{x^n}}{(\sin{x})^m}$(其中$m>0,n>0$為常數)
    \vspace{\stretch{2}}
    \newpage %---------------------------------------------------
    \bonuspart
    \unimportant $\displaystyle \lim_{n \to \infty} \left( \frac{n}{n^2+1}+\frac{n}{n^2+2}+\dots+\frac{n}{n^2+n} \right)$ \hint{放縮法}
    \vspace{\stretch{2}}
    \part[1]
    $\displaystyle \lim_{n \to \infty} \left( \frac{1}{n^2}+\frac{2}{n^2}+\dots+\frac{n}{n^2} \right)$ \hint{定積分的定義或者暴力求解}
    \vspace{\stretch{2}}
    \bonuspart
    \unimportant $\displaystyle \lim_{n \to \infty} \left( \frac{n}{n^2+1^2}+\frac{n}{n^2+2^2}+\dots+\frac{n}{n^2+n^2} \right)$ \hint{定積分的定義}
    \vspace{\stretch{2}}
    \bonuspart
    \unimportant 設$a_1>0$,$a_{n+1}=\ln{(1+a_n)}$。證明:$\displaystyle \lim_{n \to \infty} a_n$存在,並求此極限。
    \vspace{\stretch{2}}
    \newpage %---------------------------------------------------
    \part[1]
    $\displaystyle \lim_{x \to 0} x\sin{\frac{1}{x}}$
    \vspace{\stretch{2}}
    \part[1]
    $\displaystyle \lim_{x \to \infty} \frac{\sin{x}}{x}$
    \vspace{\stretch{2}}
    \part[1]
    $\displaystyle \lim_{x \to 0} \frac{\sin{x}}{x}$
    \vspace{\stretch{2}}
    \part[1]
    $\displaystyle \lim_{x \to \infty} x\sin{\frac{1}{x}}$
    \vspace{\stretch{2}}
    \part[1]
    \important $\displaystyle \lim_{x \to 0} x^2\sin{\frac{1}{x}}$
    \vspace{\stretch{2}}
    \newpage %---------------------------------------------------
    \part[1]
    \important $\displaystyle \lim_{x \to 0} \frac{\tan^2{3x}}{x \sin{2x}}$
    \vspace{\stretch{2}}
    \part[1]
    \important $\displaystyle \lim_{x \to 0} \frac{x \ln{(1+x)}}{1-\cos{x}}$
    \vspace{\stretch{2}}
    \part[1]
    \important $\displaystyle \lim_{x \to 0} \frac{\ln{(\cos{x})}}{x^2}$
    \vspace{\stretch{2}}
    \part[1]
    \important $\displaystyle \lim_{x \to 0} \frac{\tan x -\sin x}{x \sin ^2 x}$
    \vspace{\stretch{2}}
    \newpage %---------------------------------------------------
    \part[1]
    \important $\displaystyle \lim_{x \to \infty} \left( 1+\frac{4}{x} \right)^{x}$
    \vspace{\stretch{2}}
    \part[1]
    \important $\displaystyle \lim_{x \to 0} (\cos{x})^{\frac{1}{x^2}}$
    \vspace{\stretch{2}}
    \part[1]
    \important $\displaystyle \lim_{x \to 0} \left( \frac{2x+3}{2x+1} \right)^{x+1}$
    \vspace{\stretch{2}}
    \part[1]
    \important $\displaystyle \lim_{n \to \infty} \left( \frac{n-2}{n} \right)^{3n}$
    \vspace{\stretch{2}}
    \part[1]
    \important $\displaystyle \lim_{x \to 0} \frac{1}{x} \left[ \left( \frac{4-\cos{x}}{3} \right)^{\frac{1}{x}} -1 \right]$
    \vspace{\stretch{2}}
    \newpage %---------------------------------------------------
    \bonuspart
    \unimportant $\displaystyle \lim_{x \to +\infty} \frac{e^x}{\left( 1+\frac{1}{x} \right)^{x^2}}$
    \vspace{\stretch{2}}
    \part[1]
    $\displaystyle \lim_{x \to 0} \frac{e^x-e^{-x}}{\sin{x}}$
    \vspace{\stretch{2}}
    \bonuspart
    \unimportant $\displaystyle \lim_{x \to 0} \frac{\cos{(\sin{x})}-\cos{x}}{x^4}$ \hint{拉格朗日中值定理}
    \vspace{\stretch{2}}
    \part[1]
    $\displaystyle \lim_{x \to 0} \frac{x^2-\int_0^{x^2}\cos{t^2} \d t}{x \sin^9{x}}$ \hint{洛必達法則}
    \vspace{\stretch{2}}
\end{parts}

\end{questions}

\newpage %---------------------------------------------------

\subsection{函數的連續性與間斷點}

\begin{questions}

\question
\important 當函數在一點的左右極限存在並且相等,則說函數在這個點有\textbf{極限},這個極限是它的左右極限值。當函數在一點的極限值等於函數值,則說函數在這一點\textbf{連續}。否則,則說函數在這一點\textbf{間斷},這一點是函數的\textbf{間斷點}。間斷點的分類如下:
\begin{itemize}
    \item
    左右極限\textbf{存在且相等},為\textbf{第一類間斷點}中的\textbf{可去間斷點};
    \item
    左右極限\textbf{存在但不相等},為\textbf{第一類間斷點}中的\textbf{跳躍間斷點};
    \item
    左右極限\textbf{有至少一個不存在},為\textbf{第二類間斷點}。
\end{itemize}
\vspace{\stretch{1}}

\question[1]
求$\displaystyle f(x)=\frac{x}{x}$、$\displaystyle g(x)=\frac{|x|}{x}$當$x \to 0$時的左右極限,並說明它們當$x \to 0$時的極限是否存在。
\vspace{\stretch{2}}

\newpage %---------------------------------------------------

\question[1]
設$\displaystyle f(x)=\frac{e^{\frac{1}{x}}-1}{e^{\frac{1}{x}}+1}$,則$x=0$是$f(x)$的
\begin{choices}
    \choice 可去間斷點
    \choice 跳躍間斷點
    \choice 第二類間斷點
    \choice 連續點
\end{choices}
\vspace{\stretch{1}}

\question[1]
設$\displaystyle f(x)=\frac{1-e^{\frac{1}{x-1}}}{1+e^{\frac{1}{x-1}}} \cdot e^{\frac{1}{x}}$,求$f(x)$的間斷點並分類。
\vspace{\stretch{2}}

\question[1]
設$f(x)=\begin{cases} 
    x &,0<x<1 \\ 
    2 &,x=1\\ 
    2+x &,1<x<2 
\end{cases}
$,則$x=1$為函數$f(x)$的\fillin[Black]間斷點。
\vspace{\stretch{1}}

\question[1]
\important 設$f(x)=\begin{cases} 
    x+1 &,x \leq 0 \\ 
    x-2 &,x>0
\end{cases}
$,則$x=0$為函數$f(x)$的\fillin[Black]間斷點。
\vspace{\stretch{1}}

\end{questions}

\newpage %---------------------------------------------------

\subsection{漸近線}

\begin{questions}

\question
\begin{itemize}
    \item \important 若$\displaystyle \lim_{x \to a}f(x)=\infty$,或$f(a-0)=\infty$或$f(a+0)=\infty$,稱$x=a$為$L: y=f(x)$的\textbf{鉛直漸近線};
    \item \important 若$\displaystyle \lim_{x \to \infty}f(x)=A$,稱$y=A$為$L: y=f(x)$的\textbf{水平漸近線};
    \item \unimportant 若$\displaystyle \lim_{x \to \infty}\frac{f(x)}{x}=a(a \neq 0)$,$\displaystyle \lim_{x \to \infty} \left[ f(x)-ax \right]=b$,稱$y=ax+b$為$L: y=f(x)$的\textbf{斜漸近線}。
\end{itemize}
\vspace{\stretch{1}}

\question[1]
曲線$\displaystyle y=\frac{x+1}{x^2-1}$的鉛直漸近線為
\vspace{\stretch{2}}

\question[1]
曲線$\displaystyle y=\frac{x+3}{2x-1}$的水平漸近線為
\vspace{\stretch{2}}

\bonusquestion
\unimportant 求曲線$\displaystyle y=\frac{2}{x}+\ln{(1+e^x)}$的全部漸近線。
\vspace{\stretch{2}}

\end{questions}

\newpage %---------------------------------------------------

\subsection{無窮小與無窮大}

\begin{questions}

\question
對於$x \to a$時的兩個無窮小$f(x)$和$g(x)$,\begin{itemize}
    \item 若$\displaystyle \lim_{x \to a} \frac{f(x)}{g(x)}=\infty$,$g(x)$是$f(x)$的\textbf{高階無窮小},$f(x)$是$g(x)$的\textbf{低階無窮小};
    \item 若$\displaystyle \lim_{x \to a} \frac{f(x)}{g(x)}=0$,$f(x)$是$g(x)$的\textbf{高階無窮小},$g(x)$是$f(x)$的\textbf{低階無窮小};
    \item 若$\displaystyle \lim_{x \to a} \frac{f(x)}{g(x)}=C(C\text{為常數})$,$f(x)$與$g(x)$為\textbf{同階無窮小};
    \item \important 若$\displaystyle \lim_{x \to a} \frac{f(x)}{g(x)}=1$,$f(x)$與$g(x)$為\textbf{同階無窮小}中的一個特殊情況,\textbf{等價無窮小}。
\end{itemize}
\vspace{\stretch{1}}

\question[1]
當$x \to 0$時,下列變量中是無窮小量的有
\begin{choices}
    \choice $\displaystyle \sin{\frac{1}{x}}$
    \choice $\displaystyle \frac{\sin{x}}{x}$
    \choice $2^{-x}-1$
    \choice $\ln{|x|}$
\end{choices}
\vspace{\stretch{1}}

\question[1]
函數$\displaystyle \frac{1+2x^3}{x^2}$為當$x \to 0$時的無窮\fillin[Black]量。
\vspace{\stretch{1}}

\newpage %---------------------------------------------------

\question[1]
無窮多個無窮小量之和
\begin{choices}
    \choice 必是無窮小量
    \choice 必是無窮大量
    \choice 必是有界量
    \choice 可能是無窮小,可能是無窮大,也有可能是有界量
\end{choices}
\vspace{\stretch{1}}

\question[1]
若$\displaystyle \lim_{x \to x_0}f(x)=\infty$,$\displaystyle \lim_{x \to x_0}g(x)=\infty$,則下列正確的是
\begin{choices}
    \choice $\displaystyle \lim_{x \to x_0}f(x)+g(x)=\infty$
    \choice $\displaystyle \lim_{x \to x_0}f(x)-g(x)=\infty$
    \choice $\displaystyle \lim_{x \to x_0}\frac{1}{f(x)+g(x)}=0$
    \choice $\displaystyle \lim_{x \to x_0}kf(x)=\infty$
\end{choices}
\vspace{\stretch{1}}

\question[1]
$\sin{x}$與$x$當$x \to 0$時為\fillin[Black]無窮小。
\vspace{\stretch{1}}

\question[1]
$1-\cos{x}$與$x^2$當$x \to 0$時為\fillin[Black]無窮小。
\vspace{\stretch{1}}

\newpage %---------------------------------------------------

\question[1]
當$x \to 0$時,與$\ln{(1+x^2)}$為同階無窮小但不為等價無窮小的是
\begin{choices}
    \choice $\sin{x}\tan^2{x}$
    \choice $1-\cos{x}$
    \choice $\displaystyle \left( 1+\frac{1}{2}x^2 \right)^2-1$
    \choice $x \sin{x}$
\end{choices}
\vspace{\stretch{1}}

\question[1]
當$x \to 0$時,$f(x)=x-\sin ax$與$g(x)=x^2 \ln (1-bx)$是等價無窮小量,則
\begin{choices}
    \choice $\displaystyle a=1, b=-\frac{1}{6}$
    \choice $\displaystyle a=1, b=\frac{1}{6}$
    \choice $\displaystyle a=-1, b=-\frac{1}{6}$
    \choice $\displaystyle a=-1, b=\frac{1}{6}$
\end{choices}
\vspace{\stretch{1}}

\bonusquestion
\unimportant 設$\displaystyle \lim_{x \to 0} \frac{\ln{\left[ 1+\dfrac{f(x)}{x^2} \right]}}{\arctan{x}}=1$,求$\displaystyle \lim_{x \to 0} \frac{f(x)}{(1-\cos{x})\tan{x}}$。\hint{等價無窮小代換}
\vspace{\stretch{2}}

\end{questions}

\newpage %---------------------------------------------------

\subsection{函數的性質}

\begin{questions}

\question[1]
下列函數在$(-\infty,+\infty)$內無界的是
\begin{choices}
    \choice $\displaystyle y=\frac{1}{1+x^2}$
    \choice $\displaystyle y=\arctan{x}$
    \choice $\displaystyle y=\sin{x}+\cos{x}$
    \choice $\displaystyle y=x \sin{x}$
\end{choices}
\vspace{\stretch{1}}

\question[1]
求下列函數的定義域。\\
$y=\ln{(x+5)}$
\vspace{\stretch{2}}

\end{questions}

\newpage %---------------------------------------------------

\section{一元函數微分學}

\subsection{導數的定義與複合函數求導}

\begin{questions}

\question
\unimportant \textbf{導數}指的是一個函數在某一瞬間的變化率。比如說位移的導數是速度,速度的導數是加速度。例如函數$f(x)$在$x$時點的變化率$f'(x)$可以寫成$\displaystyle f'(x)=\lim_{\Delta x \to 0} \frac{f(x+\Delta x)-f(x)}{\Delta x}=\lim_{h \to x} \frac{f(h)-f(x)}{h-x}$。關於導數的四則運算和鏈式法則的內容高中都學過,這裡就不細講了。
\vspace{\stretch{1}}

\question
\important 默寫常用的求導基本公式。
\begin{parts}
    \part
    $\displaystyle \left( C \right)'=$
    \vspace{\stretch{2}}
    \part
    $\displaystyle \left( x^a \right)'=$
    \vspace{\stretch{2}}
    \part
    $\displaystyle \left( \sqrt{x} \right)'=$
    \vspace{\stretch{2}}
    \part
    $\displaystyle \left( \frac{1}{x} \right)'=$
    \vspace{\stretch{2}}
    \part
    $\displaystyle \left( a^x \right)'=(a>0,a\neq1)$
    \vspace{\stretch{2}}
    \part
    $\displaystyle \left( e^x \right)'=$
    \vspace{\stretch{2}}
    \part
    $\displaystyle \left( \log_a x \right)'=$
    \vspace{\stretch{2}}
    \part
    $\displaystyle \left( \ln x \right)'=$
    \vspace{\stretch{2}}

\newpage %---------------------------------------------------
    
    \part
    $\displaystyle \left( \sin x \right)'=$
    \vspace{\stretch{2}}
    \part
    $\displaystyle \left( \cos x \right)'=$
    \vspace{\stretch{2}}
    \part
    $\displaystyle \left( \tan x \right)'=$
    \vspace{\stretch{2}}
    \part
    $\displaystyle \left( \cot x \right)'=$
    \vspace{\stretch{2}}
    \part
    $\displaystyle \left( \sec x \right)'=$
    \vspace{\stretch{2}}
    \part
    $\displaystyle \left( \csc x \right)'=$
    \vspace{\stretch{2}}
    \part
    $\displaystyle \left( \arcsin x \right)'=$
    \vspace{\stretch{2}}
    \part
    $\displaystyle \left( \arccos x \right)'=$
    \vspace{\stretch{2}}
    \part
    $\displaystyle \left( \arctan x \right)'=$
    \vspace{\stretch{2}}
    \part
    $\displaystyle \left( \arccot x \right)'=$
    \vspace{\stretch{2}}
\end{parts}

\newpage %---------------------------------------------------

\bonusquestion
\unimportant 設$\displaystyle f(x)=\begin{cases}
    \ln (1+ax) +b &, x>0\\
    e^{2x} &, x \leq 0
\end{cases}$,若$f'(0)$存在,求$a$、$b$的值。
\vspace{\stretch{2}}

\bonusquestion
\unimportant 已知函數$f(x)$在$x=1$附近連續。通過以下哪一個選項的條件可以推斷出$f'(1)=-2$?
\begin{choices}
    \choice $\displaystyle \lim_{x \to 0} \frac{f(1+x)-f(1)}{2x}=-2$
    \choice $\displaystyle \lim_{x \to 0} \frac{f(1+x^3)-f(1)}{\sin^2 x}=-2$
    \choice $\displaystyle \lim_{x \to 0} \frac{f(\cos x)-f(1)}{x^2}=1$
    \choice $\displaystyle \lim_{x \to 0} \frac{f(1-x^3)-f(1)}{\sin x-x}=-12$
\end{choices}
\vspace{\stretch{1}}

\question[1]
設函數$f(x)=|\sin x|$,則$f(x)$在$x=0$處
\begin{choices}
    \choice 不連續
    \choice 連續,但不可導
    \choice 可導,但不連續
    \choice 可導,且導數也連續
\end{choices}
\vspace{\stretch{1}}

\newpage %---------------------------------------------------

\question
求下列函數的導數。

\begin{parts}

    \part[1]
    $\displaystyle y=e^{\arctan \sqrt{x}}$
    \vspace{\stretch{2}}
    
    \part[1]
    $\displaystyle y=\ln \tan \frac{x}{2}$
    \vspace{\stretch{2}}
    
    \part[1]
    $\displaystyle s=a \cos^2 (2\omega t + \phi)$
    \vspace{\stretch{2}}

\end{parts}

\newpage %---------------------------------------------------

\question[1]
\important 函數$\displaystyle \frac{\ln x}{x}$是$f(x)$的一個原函數,則$f(x)=$\fillin[Blank]。
\vspace{\stretch{1}}

\question[1]
\important 函數$y=e^{2-3x}$,則$y^{(n)}=$\fillin[Blank]。
\vspace{\stretch{1}}

\question[1]
設$y=xe^x$,則$y^{(n)}=$
\begin{choices}
    \choice $\displaystyle e^x(x+n)$
    \choice $\displaystyle e^x(x-n)$
    \choice $\displaystyle 2e^x(x+n)$
    \choice $\displaystyle xe^{nx}$
\end{choices}
\vspace{\stretch{1}}

\question[1]
求曲線$y=\cos x$在點$\displaystyle \left( \frac{\pi}{3}, \frac{1}{2} \right)$處的切線和法線方程。
\vspace{\stretch{2}}

\end{questions}

\newpage %---------------------------------------------------

\subsection{隱函數求導}

\begin{questions}

\question
像是由方程$F(x,y)=0$確定的$y$是$x$的函數我們叫它隱函數。隱函數的求導主要有兩種方法\begin{itemize}
    \item 一個是把函數左右兩邊對兩邊求導,再通過移項等操作將$y'$移到等號一邊,其他項移到等號另一邊。需要注意,在求導時要將$x$看成$y$的函數,即在求導到關於$y$的多項式時鏈式法則需要一直波及到$x$。
    \item \important 另一個方法是帶公式$\displaystyle \frac{\d y}{\d x} = -\frac{F_x'}{F_y'}$。其中,$F_x'$為$F(x,y)$對$x$的偏導數,$F_y'$為$F(x,y)$對$y$的偏導數。在求函數對一個元的偏導數的時候,要將另一個元看作常數。
\end{itemize}
\vspace{\stretch{1}}

\question[1]
求下列方程所確定的隱函數$y$的導數$\displaystyle \frac{\d y}{\d x}$

$xy=e^{x+y}$
\vspace{\stretch{2}}

\question[1]
\important 設$x+y-xe^y=0$,求$\d y$。
\vspace{\stretch{2}}

\end{questions}

\newpage %---------------------------------------------------

\subsection{參數方程確定的函數求導}

\begin{questions}

\question
\important 對於$\displaystyle \begin{cases}
y=y(t)\\
x=x(t) 
\end{cases}$這樣的函數,我們稱為參數方程確定的函數。它的導數可以這樣求:
\begin{align*}
    \frac{\d y}{\d x} &= \frac{\d y / \d t}{\d x / \d t}=\frac{y'(t)}{x'(t)}\\
    \frac{\d^2 y}{\d x^2} &= \frac{\d \left( \frac{\d y}{\d x} \right) / \d t}{\d x / \d t}=\frac{\left[ 
\frac{y'(t)}{x'(t)} \right]'}{x'(t)}=\frac{y''(t)x'(t)-y'(t)x''(t)}{[x'(t)]^3}
\end{align*}
\vspace{\stretch{1}}

\question[1]
\important 設$\begin{cases}
x=\sin t+3\\
y=t-\cos 2t
\end{cases}$,則$\displaystyle \frac{\d ^2 y}{\d x^2}=$\fillin[Black]。
\vspace{\stretch{1}}

\question[1]
設由方程$\begin{cases}
x=a(t-\sin t)\\
y=a(1-\cos t)
\end{cases}$所確定的函數為$y=y(x)$,則在$\displaystyle t=\frac{\pi}{2}$處導數為
\begin{choices}
    \choice $-1$
    \choice $1$
    \choice $0$
    \choice $\displaystyle -\frac{1}{2}$
\end{choices}
\vspace{\stretch{1}}

\question[1]
求下列方程所確定的隱函數$y$的二階導數$\displaystyle \frac{\d^2 y}{\d x^2}$

$\begin{cases}
x=\ln(1+t^2)\\
y=t-\arctan t
\end{cases}$
\vspace{\stretch{2}}

\end{questions}

\newpage %---------------------------------------------------

\subsection{微分中值定理}

\begin{questions}

\question
\begin{description}
    \item[零點定理:] 若函數$f(x)$在$[a,b]$連續,$f(a)f(b)<0$,則一定存在$\xi \in (a,b)$使得$f(\xi)=0$。
    \item[\important 羅爾定理:] 若函數$f(x)$在$[a,b]$連續,在$(a,b)$可導,$f(a)=f(b)$,則一定存在$\xi \in (a,b)$使得$f'(\xi)=0$。
    \item[拉格朗日中值定理:] 若函數$f(x)$在$[a,b]$連續,在$(a,b)$可導,則一定存在$\xi \in (a,b)$使得$\displaystyle f'(\xi)=\frac{f(b)-f(a)}{b-a}$,或者寫作$f(b)-f(a)=f'(\xi)(b-a)$。
    \item[柯西中值定理:] 若函數$f(x)$和$g(x)$在$[a,b]$連續,在$(a,b)$可導,$g'(x) \neq 0$,則一定存在$\xi \in (a,b)$使得$\displaystyle \frac{f'(\xi)}{g'(\xi)}=\frac{f(b)-f(a)}{g(b)-g(a)}$。
\end{description}
\vspace{\stretch{1}}

\question
函數$f(x)$在$[a,b]$連續,在$(a,b)$可導,$f(a)=f(b)=0$,並且$a>0, b>0$。嘗試證明:
\begin{parts}
    \part[1]
    在區間$(a,b)$內存在一點$\xi$使得$f(\xi)+\xi f'(\xi)=0$。\hint{“$\xi$”讀作“克西”}
    \vspace{\stretch{2}}
    \part[1]
    在區間$(a,b)$內存在一點$\epsilon$使得$5f(\epsilon)+\epsilon f'(\epsilon)=0$。\hint{“$\epsilon$”讀作“艾普西隆”}
    \vspace{\stretch{2}}
\newpage %---------------------------------------------------
    \part[1]
    \important 在區間$(a,b)$內存在一點$\zeta$使得$f(\zeta)+f'(\zeta)=0$。\hint{“$\zeta$”讀作“澤塔”}
    \vspace{\stretch{2}}
    \part[1]
    \important 在區間$(a,b)$內存在一點$\eta$使得$5f(\eta)+f'(\eta)=0$。\hint{“$\eta$”讀作“伊塔”}
    \vspace{\stretch{2}}
    \part[1]
    \important 在區間$(a,b)$內存在一點$\mu$使得$2\mu f(\mu)+f'(\mu)=0$。\hint{“$\mu$”讀作“謬”}
    \vspace{\stretch{2}}
\end{parts}

\end{questions}

\newpage %---------------------------------------------------

\subsection{單調性與極值、凹凸性與拐點、函數作圖}

\begin{questions}

\question 
\important 一階導大於0的是\textbf{增區間},一階導小於0的是\textbf{減區間},由增區間往減區間轉變的是\textbf{極大值點},由減區間往增區間轉變的是\textbf{極小值點}。二階導大於0的是\textbf{凹區間},二階導小於0的是\textbf{凸區間},一階導等於0二階導大於0的是\textbf{極小值點},一階導等於0二階導小於0的是\textbf{極大值點},凹區間和凸區間交接處是\textbf{拐點}。
\vspace{\stretch{1}}

\question[1]
若$(x_0, f(x_0))$為連續曲線$y=f(x)$上的凹弧與凸弧分界點,則
\begin{choices}
    \choice $(x_0, f(x_0))$必定為曲線的拐點
    \choice $(x_0, f(x_0))$必定為曲線的駐點
    \choice $x_0$為$f(x)$的極值點
    \choice $x_0$必定不是$f(x)$的極值點
\end{choices}
\vspace{\stretch{1}}

\question[1]
下列結論正確的是
\begin{choices}
    \choice 駐點一定是極值點
    \choice 可導函數的極值點一定是駐點
    \choice 函數的不可導點一定是極值點
    \choice 函數的極大值一定大於極小值
\end{choices}
\vspace{\stretch{1}}

\newpage %---------------------------------------------------

\question[1]
\important \textbf{\underline{在同一表中}}討論$y=1+3x^2-x^3$的單調性、極值、凹凸性、拐點。
\vspace{\stretch{2}}

\end{questions}

\newpage %---------------------------------------------------

\section{一元函數積分學}

\subsection{積分計算}

\begin{questions}

\question \important 不定積分可以出很難的題。\textbf{實在做不出來就算了。}
\vspace{\stretch{1}}

\question \important 默寫常用的不定積分基本公式:
\begin{parts}
    \part
    $\displaystyle \int k \d x = (k\text{為常數})$
    \vspace{\stretch{2}}
    \part
    $\displaystyle \int x^a \d x = (a \neq 1)$
    \vspace{\stretch{2}}
    \part
    $\displaystyle \int \frac{1}{x} \d x = (x \neq 0)$
    \vspace{\stretch{2}}
    \part
    $\displaystyle \int a^x \d x = (a>0, a \neq 1)$
    \vspace{\stretch{2}}
    \part
    $\displaystyle \int e^x \d x = $
    \vspace{\stretch{2}}
    \newpage %---------------------------------------------------
    \part
    $\displaystyle \int \sin x \d x = $
    \vspace{\stretch{2}}
    \part
    $\displaystyle \int \cos x \d x = $
    \vspace{\stretch{2}}
    \part
    $\displaystyle \int \tan x \d x = $
    \vspace{\stretch{2}}
    \part
    $\displaystyle \int \cot x \d x = $
    \vspace{\stretch{2}}
    \part
    $\displaystyle \int \sec x \d x = $
    \vspace{\stretch{2}}
    \part
    $\displaystyle \int \csc x \d x = $
    \vspace{\stretch{2}}
    \part
    $\displaystyle \int \sec^2 x \d x = $
    \vspace{\stretch{2}}
    \part
    $\displaystyle \int \csc^2 x \d x = $
    \vspace{\stretch{2}}
    \newpage %---------------------------------------------------
    \part
    $\displaystyle \int \frac{1}{\sqrt{1-x^2}} \d x = $
    \vspace{\stretch{2}}
    \part
    $\displaystyle \int \frac{1}{\sqrt{a^2-x^2}} \d x = $
    \vspace{\stretch{2}}
    \part
    $\displaystyle \int \frac{1}{1+x^2} \d x = $
    \vspace{\stretch{2}}
    \part
    $\displaystyle \int \frac{1}{a^2+x^2} \d x = $
    \vspace{\stretch{2}}
    \part
    $\displaystyle \int \sec x \tan x \d x = $
    \vspace{\stretch{2}}
    \part
    $\displaystyle \int \csc x \cot x \d x = $
    \vspace{\stretch{2}}
    \newpage %---------------------------------------------------
    \part
    $\displaystyle \int \frac{1}{x^2-a^2} \d x = $
    \vspace{\stretch{2}}
    \part
    $\displaystyle \int \frac{1}{\sqrt{x^2+a^2}} \d x = $
    \vspace{\stretch{2}}
    \part
    $\displaystyle \int \frac{1}{\sqrt{x^2-a^2}} \d x = $
    \vspace{\stretch{2}}
    \part
    $\displaystyle \int \sqrt{a^2-x^2} \d x = $
    \vspace{\stretch{2}}
\end{parts}

\newpage %---------------------------------------------------

\question[1]
已知函數$f(x)=\begin{cases}
    2(x-1) &,x<1\\
    \ln x &, x\geq 1
\end{cases}$,則$f(x)$的一個原函數是
\begin{choices}
    \choice $\displaystyle F(x)=\begin{cases}
        (x-1)^2 &,x<1\\
        x(\ln x -1) &, x\geq 1
    \end{cases}$
    \choice $\displaystyle F(x)=\begin{cases}
        (x-1)^2 &,x<1\\
        x(\ln x +1)-1 &, x\geq 1
    \end{cases}$
    \choice $\displaystyle F(x)=\begin{cases}
        (x-1)^2 &,x<1\\
        x(\ln x +1)+1 &, x\geq 1
    \end{cases}$
    \choice $\displaystyle F(x)=\begin{cases}
        (x-1)^2 &,x<1\\
        x(\ln x -1)+1 &, x\geq 1
    \end{cases}$
\end{choices}
\vspace{\stretch{1}}

\question[1]
設$\displaystyle I=\int_0^{\frac{\pi}{4}} \ln (\sin x) \d x$,$\displaystyle J=\int_0^{\frac{\pi}{4}} \ln (\cot x) \d x$,$\displaystyle K=\int_0^{\frac{\pi}{4}} \ln (\cos x) \d x$,則$I$、$J$、$K$的大小關係為
\begin{choices}
    \choice $I<J<K$
    \choice $I<K<J$
    \choice $J<I<K$
    \choice $K<J<I$
\end{choices}
\vspace{\stretch{1}}

\newpage %---------------------------------------------------

\question
計算下列積分:
\begin{parts}

    \part[1]
    $\displaystyle \int 3^x e^x \d x$
    \vspace{\stretch{2}}

    \part[1]
    \important $\displaystyle \int (3-2x)^3 \d x$
    \vspace{\stretch{2}}

    \part[1]
    \important $\displaystyle \int \cos^2 3x \d x$
    \vspace{\stretch{2}}

    \part[1]
    \important $\displaystyle \int x \cos (x^2) \d x$
    \vspace{\stretch{2}}

    \part[1]
    \important $\displaystyle \int_{0}^{\frac{\pi}{2}} \sin x \cos ^3 x \d x$
    \vspace{\stretch{2}}

    \newpage %---------------------------------------------------

    \part[1]
    \important $\displaystyle \int_{1}^{4} \frac{1}{1+\sqrt{x}} \d x$ \hint{第二類換元法}
    \vspace{\stretch{2}}

    \part[1]
    \important $\displaystyle \int \frac{\sqrt{x}}{\sqrt{x}-1} \d x$ \hint{第二類換元法}
    \vspace{\stretch{2}}
    
    \bonuspart
    \unimportant $\displaystyle \int \sqrt{9-x^2} \d x$ \hint{第二類換元法}
    \vspace{\stretch{2}}

    \newpage %---------------------------------------------------

    \part[1]
    \important $\displaystyle \int_{-\pi}^{\pi} x^3 \cos ^2 2x \d x$ \hint{定積分的幾何意義}
    \vspace{\stretch{2}}

    \part[1]
    \important $\displaystyle \int_{-3}^{3} \sqrt{9-x^2} \d x$ \hint{定積分的幾何意義}
    \vspace{\stretch{2}}

    \part[1]
    \important $\displaystyle \int_{-3}^{0} (x+\sqrt{9-x^2}) \d x$ \hint{定積分的幾何意義}
    \vspace{\stretch{2}}
    
    \newpage %---------------------------------------------------

    \part[1]
    \important $\displaystyle \int \arcsin x \d x$
    \vspace{\stretch{2}}

    \part[1]
    \important $\displaystyle \int \frac{\ln x}{\sqrt{x}} \d x$
    \vspace{\stretch{2}}

    \part[1]
    \important $\displaystyle \int x \ln (x+1) \d x$
    \vspace{\stretch{2}}

    \part[1]
    \important $\displaystyle \int x^2 \ln 2x \d x$
    \vspace{\stretch{2}}

    \part[1]
    \important $\displaystyle \int_{0}^{1} x e^{-x} \d x$
    \vspace{\stretch{2}}

    \newpage %---------------------------------------------------

    \part[1]
    \important \unimportant $\displaystyle \int_{1}^{+\infty} \left( 3^{-x} + \frac{1}{x^4} \right) \d x$
    \vspace{\stretch{2}}
    
\end{parts}

\question[1]
下列積分中的反常積分為
\begin{choices}
    \choice $\displaystyle \int_0^1 \frac{1}{2-x} \d x$
    \choice $\displaystyle \int_0^1 \frac{1}{2+x} \d x$
    \choice $\displaystyle \int_0^2 \frac{1}{1+x^2} \d x$
    \choice $\displaystyle \int_0^2 \frac{1}{1-x^2} \d x$
\end{choices}
\vspace{\stretch{1}}

\question[1]
把有理函數$\displaystyle f(x)=\frac{1}{(x^2+1)(x^2+x+1)}$化為部分分式的和,需要拆項為
\begin{choices}
    \choice $\displaystyle \frac{C}{x^2+1}$和$\displaystyle \frac{D}{x^2+x+1}$
    \choice $\displaystyle \frac{Ax+C}{x^2+1}$和$\displaystyle \frac{D}{x^2+x+1}$
    \choice $\displaystyle \frac{Ax+C}{x^2+1}$和$\displaystyle \frac{Bx+D}{x^2+x+1}$
    \choice $\displaystyle \frac{C}{x^2+1}$和$\displaystyle \frac{Bx+D}{x^2+x+1}$
\end{choices}
\vspace{\stretch{1}}

\end{questions}

\newpage %---------------------------------------------------

\subsection{定積分應用}

\begin{questions}

\question
\important 直角座標下由$y=f(x)$、$y=g(x)$、$x=a$、$x=b$圍成圖形的面積:
\begin{align*}
    A=\int_a^b \left| f(x)-g(x) \right| \d x
\end{align*}

\unimportant 極座標下由$r=r_1(\theta)$、$r=r_2(\theta)$,$r_1(\theta) \leq r_2(\theta)$,$\alpha \leq \theta \leq \beta$圍成圖形的面積:
\begin{align*}
    A=\frac{1}{2}\int_{\alpha}^{\beta} \left[ r_2^2(\theta) - r_1^2(\theta) \right] \d x
\end{align*}
\vspace{\stretch{1}}

\question[1]
求直線$y=2x+3$與拋物線$y=x^2$所圍成圖形的面積。
\vspace{\stretch{2}}

\bonusquestion
\unimportant 求$r=2a \cos \theta$所圍成圖形的面積。
\vspace{\stretch{2}}

\newpage %---------------------------------------------------

\question
有一個$y=x^2$與$y=1$及$x=0$所圍成在第一象限內的圖形。
\begin{parts}

    \part[1]
    求其繞$x$軸旋轉一週得到的旋轉體的體積。
    \vspace{\stretch{2}}
    \part[1]
    求其繞$y$軸旋轉一週得到的旋轉體的體積。
    \vspace{\stretch{2}}
    \part
    \unimportant 求其繞直線$y=x+1$旋轉一週得到的旋轉體的體積。
    \vspace{\stretch{2}}
    
\end{parts}

\newpage %---------------------------------------------------

\question
\important 本題有兩小問。
\begin{parts}

    \part[1]
    求曲線$y=x^2$,直線$y=0$及$x=3$所圍成的圖形的面積。
    \vspace{\stretch{2}}

    \part[1]
    將上述平面圖形繞$y$軸旋轉一週,求所得立體的體積。
    \vspace{\stretch{2}}

\end{parts}

\end{questions}

\newpage %---------------------------------------------------

\section{常微分方程}

\subsection{微分方程的階數}

\begin{questions}

\question[1]
微分方程$y''=2x^2+3$的階數為
\begin{choices}
    \choice 1
    \choice 2
    \choice 3
    \choice 4
\end{choices}
\vspace{\stretch{1}}

\question[1]
$xy(y')^2-yy'-x=0$為\fillin[Blank]階的微分方程。
\vspace{\stretch{1}}

\question[1]
$\displaystyle L \frac{\d^2 Q}{\d t^2} +Q^3 \frac{\d Q}{\d t} +Q=0$為\fillin[Blank]階的微分方程。
\vspace{\stretch{1}}
    
\end{questions}

\newpage %---------------------------------------------------

\subsection{一階微分方程的種類及解法}

\begin{questions}

\question
\important 一階非齊次線性微分方程$\displaystyle \frac{\d y}{\d x} +P(x)y = Q(x)$的通解公式為
\begin{align*}
    y=\left[ \int Q(x) e^{\int P(x) \d x} \d x + C \right]e^{- \int P(x) \d x}
\end{align*}
\vspace{\stretch{1}}

\question
求解下列微分方程。
\begin{parts}

    \part[1]
    $\displaystyle xy'-y \ln y = 0$
    \vspace{\stretch{2}}
    \part[1]
    $\displaystyle \frac{\d y}{\d x}=10^{x+y}$
    \vspace{\stretch{2}}
    \newpage %---------------------------------------------------
    \part[1]
    \important $\displaystyle (x^2-1)y'+2xy-\cos x=0$
    \vspace{\stretch{2}}
    \part[1]
    \important $\displaystyle y'+y \sin x = e^{\cos x}$
    \vspace{\stretch{2}}

\end{parts}
    
\end{questions}

\newpage %---------------------------------------------------

\subsection{可降階的高階微分方程求解}

\begin{questions}

\question
求解下列微分方程。
\begin{parts}

    \part[1]
    $\displaystyle y'''=\sin 2x$
    \vspace{\stretch{2}}
    \bonuspart
    \unimportant 求微分方程$\displaystyle y''+2xy'^2=0$滿足初始條件$\displaystyle y(0)=1, y'(0)=-\frac{1}{2}$的特解。
    \vspace{\stretch{2}}
    \bonuspart
    \unimportant 求微分方程$\displaystyle yy''+y'^2=y'$的通解。
    \vspace{\stretch{2}}
    \newpage %---------------------------------------------------
    \bonuspart
    \unimportant 求微分方程$\displaystyle y'''=\frac{3x^2}{1+x^3}y''$滿足初始條件$\displaystyle y(0)=0, y'(0)=1, y''(0)=4$的特解。
    \vspace{\stretch{2}}
    \part[1]
    \unimportant \important $\displaystyle y''-y'=x$
    \vspace{\stretch{2}}

\end{parts}
    
\end{questions}

\newpage %---------------------------------------------------

\subsection{高階常係數線性齊次(非齊次)微分方程}

\begin{questions}

\question
對於二階常係數齊次線性微分方程$y''+py'+qy=0$,先由其特徵方程$r^2+pr+q=0$求出解$r_1, r_2$,若
\begin{enumerate}
    \item $r_1, r_2$為兩個不相等的實根,微分方程通解為$\displaystyle y=C_1 e^{r_1 x}+C_2 e^{r_2 x}$;
    \item $r_1, r_2$為兩個相等的實根,微分方程通解為$\displaystyle y=(C_1 + C_2 x) e^{r_1 x}$;
    \item $r_1, r_2$為兩個共軛虛根$\alpha \pm \beta i$,微分方程通解為$\displaystyle y=e^{\alpha x}(C_1 \cos \beta x + C_2 \sin \beta x)$。
\end{enumerate}
\vspace{\stretch{1}}

\question
對於二階常係數非齊次線性微分方程$y''+py'+qy=P(x)e^{kx}$的特解,首先設$Q(x)=a_0+a_1x+a_2x^2+ \dots +a_nx^n$與$P(x)$次數相同,再
\begin{enumerate}
    \item 若$k$非特徵值,則特解為$y^*=Q(x)e^{kx}$;
    \item 若$k$與一個特徵值相同,則特解為$y^*=xQ(x)e^{kx}$;
    \item 若$k$與兩個特徵值相同,則特解為$y^*=x^2Q(x)e^{kx}$。
\end{enumerate}
\vspace{\stretch{1}}

\newpage %---------------------------------------------------

\question
求解下列微分方程。
\begin{parts}

    \part[1]
    $\displaystyle \frac{\d^2 y}{\d x^2}-2\frac{\d y}{\d x} +5y=0$
    \vspace{\stretch{2}}
    \part[1]
    $\displaystyle y''+6y'+9y=0$
    \vspace{\stretch{2}}
    \part[1]
    $\displaystyle y''-4y'+3=0, y|_{x=0}=6, y'|_{x=0}=10  $
    \vspace{\stretch{2}}
    \newpage %---------------------------------------------------
    \part[1]
    \important $\displaystyle y''+6y'+9y=(3x+1)e^{3x}$
    \vspace{\stretch{2}}
    \part[1]
    $\displaystyle y''+3y'+2y=3xe^{-x}$
    \vspace{\stretch{2}}

\end{parts}

\end{questions}

\end{document}
